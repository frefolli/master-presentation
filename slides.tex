\documentclass[dvipsnames]{beamer}
\usepackage{xcolor}
\usepackage{graphicx}
\usepackage{listings}

\usetheme{Madrid}
\usecolortheme{beaver}
\date[26 September 2025]{}

\title[We did it Patrick! We fixed traffic!]{Unconventional reinforcement learning on traffic lights with SUMO}
\subtitle{Master Degree in Computer Science}
% \author{Refolli~F.~865955}
\author[RF 865955]{Francesco~Refolli\\[10mm]{\small Supervisor: Prof. Giuseppe Vizzari}}
\logo{\includegraphics[height=2.5cm]{logo_unimib.pdf}}

\newcommand{\putimage}[2] {
  \begin{figure}[H]
    \centering
    \includegraphics[width=#2\linewidth]{#1}
	\end{figure}
}

\newcommand{\putimagecouple}[4] {
  \begin{figure}[!htb]
    \centering
    \begin{minipage}{0.45\linewidth}
      \centering
      \includegraphics[width=\linewidth]{#1}
      \caption{#2}
    \end{minipage}
    \hspace{0.25cm}
    \begin{minipage}{0.45\linewidth}
      \centering
      \includegraphics[width=\linewidth]{#3}
      \caption{#4}
    \end{minipage}
  \end{figure}
}

\begin{document}

\frame{\titlepage}

\setbeamertemplate{logo}{}

\begin{frame}
\frametitle{Outline}
\tableofcontents
\end{frame}

\section{Introduction}

\begin{frame}
  \frametitle{Problem statement}

  This thesis dealt with the Traffic Light Control problem (TCL) applying uncommon reinforcement learning techniques.
  In particular, the following research objective were pursuit in the performed experiments:

  \begin{itemize}
    \item \textcolor{SeaGreen}{Evaluating the effectiveness of Curriculum Learning}
    \item \textcolor{SeaGreen}{Evaluating the effectiveness of Multi Agent Learning}
    \item \textcolor{CadetBlue}{Evaluating the effectiveness of Self-Adaptive agents}
    \item \textcolor{CadetBlue}{Comparing observation/reward functions}
    \item \textcolor{CadetBlue}{Comparing tabular and deep learning models}
    \item \textcolor{CadetBlue}{Comparing Reinforcement Learning with existing solutions}
    \item \textcolor{CadetBlue}{Evaluating the role of hyperparameters}
  \end{itemize}
\end{frame}

\begin{frame}
  \begin{columns}
    \begin{column}{0.4\textwidth}
      \begin{figure}
        \centering
        \includegraphics[width=1.0\textwidth]{figures/sumo-logo.png}
      \end{figure}
      \begin{itemize}
        \item Free and Open Source microscopic traffic simulator
        \item Developed at German Aerospace Center (DLR)
        \item Multimodal: cars, trams, bikes, pedestrians ...
        \item Highly customizable
      \end{itemize}
    \end{column}
    \begin{column}{0.6\textwidth}
      \begin{figure}
        \centering
        \includegraphics[width=1.0\textwidth]{figures/sumo-example.png}
      \end{figure}
      \begin{figure}
        \centering
        \includegraphics[width=1.0\textwidth]{figures/sumo-example2.png}
      \end{figure}
    \end{column}
  \end{columns}
\end{frame}

\begin{frame}
\frametitle{SUMO-RF: SUMO + Reinforcement Learning}
  A FOSS framework for Reinforcement Learning with SUMO developed as fork of \textit{LucasAlegre/sumo-rl} with a focus on modularity, flexibility and Multi Agent Learning.
  It also contains several utilities for format conversions, metrics analysis and plot, schematic-based demand generation and more.
  \begin{figure}
    \centering
    \includegraphics[width=0.8\textwidth]{figures/sumo-rf-architecture.png}
  \end{figure}
\end{frame}

\begin{frame}
\frametitle{The Agent Model}
  \begin{columns}
    \begin{column}{0.6\textwidth}
    {\small
    \begin{itemize}
      \item Each agent can control one intersection and at each step (every 5 seconds) it can choose the next phase of the intersection.
      \item Every action is automatically enforced by TrafficSignal with also an intermediate yellow phase.
      \item It receives an observation of the current condition and a reward proportional to the goodness of its behaviour.
      \item If the agent is "smart", it uses the collected data to improve itself!
    \end{itemize}}
    \end{column}
    \begin{column}{0.4\textwidth}
      \begin{figure}
        \centering
        \includegraphics[width=1.0\textwidth]{figures/sumo-rf-agent.png}
      \end{figure}
    \end{column}
  \end{columns}
\end{frame}

\begin{frame}
\frametitle{The Global State}
  \begin{figure}
    \centering
    \includegraphics[width=1.0\textwidth]{figures/sumo-rf-properties.png}
  \end{figure}
\end{frame}

\begin{frame}
\frametitle{Observing and Rewarding}
  \begin{columns}
  \begin{column}{0.4\textwidth}
    \begin{figure}
      \centering
      \includegraphics[width=1.0\textwidth]{figures/sumo-rf-observations.png}
    \end{figure}
  \end{column}
  \begin{column}{0.6\textwidth}
    \begin{figure}
      \centering
      \includegraphics[width=1.0\textwidth]{figures/sumo-rf-rewards.png}
    \end{figure}
  \end{column}
\end{columns}
\end{frame}

\section{Multi Agent Learning}

\begin{frame}
\frametitle{Multi Agent Learning}

  {\small
  No agent is isoled, they are all part of a whole and they influence each other with their own behaviour. \\
  What if an agent can \textit{sense} its surrounding area by sharing observations with neighbours? \\
  What if an agent is \textit{rewarded} for its influence on surrounding area by sharing rewards with neighbours?}

  \begin{columns}
    \begin{column}{0.3\textwidth}
      \begin{figure}
        \centering
        \includegraphics[width=0.65\textwidth]{figures/sumo-rf-tls-1.png}
      \end{figure}
    \end{column}
    \begin{column}{0.3\textwidth}
      \begin{figure}
        \centering
        \includegraphics[width=0.65\textwidth]{figures/sumo-rf-tls-2.png}
      \end{figure}
    \end{column}
    \begin{column}{0.3\textwidth}
      \begin{figure}
        \centering
        \includegraphics[width=0.65\textwidth]{figures/sumo-rf-tls-3.png}
      \end{figure}
    \end{column}
  \end{columns}
  \begin{columns}
    \begin{column}{0.5\textwidth}
      \begin{figure}
        \centering
        \includegraphics[width=0.30\textwidth]{figures/sumo-rf-tls-4.png}
      \end{figure}
    \end{column}
    \begin{column}{0.5\textwidth}
      \begin{figure}
        \centering
        \includegraphics[width=0.30\textwidth]{figures/sumo-rf-tls-5.png}
      \end{figure}
    \end{column}
  \end{columns}
\end{frame}

\begin{frame}
\frametitle{Observation sharing}

  \begin{itemize}
    \item The state $S$ of an agent is computed through two dinstict observation functions $f_{int}, f_{ext}$ meaning respectively the internal state and external state. \\
    \item The state of an agent $x$ is a concatenation of $f_{int}(x)$ and $f_{ext}(y)$ for all $y \in N(x)$, where $N$ is a function mapping an agent with its neighbours. \\
    \item By default, $f_{int}$ is the Density function which is the one that without sharing gives best results.
  \end{itemize}

  Example:
  \begin{figure}
    \centering
    \includegraphics[width=0.75\textwidth]{figures/sumo-rf-tls-triplet.png}
  \end{figure}

  
  \begin{columns}
    \begin{column}{0.4\textwidth}
      \centering
      {\footnotesize$S({TLS}_{1}) = f_{int}({TLS}_{1}) \bullet f_{ext}({TLS}_{2})$}
    \end{column}
    \begin{column}{0.6\textwidth}
      \centering
      {\footnotesize$S({TLS}_{2}) = f_{int}({TLS}_{2}) \bullet f_{ext}({TLS}_{1}) \bullet f_{ext}({TLS}_{3})$}
    \end{column}
  \end{columns}
\end{frame}

\begin{frame}
\frametitle{Reward sharing}

  \begin{itemize}
    \item The reward $R$ of an agent is computed through one single reward function $f$ representing the condition of an intersection. \\
    \item The state of an agent $x$ is a sum of $f(x)$ and $f(y)$ for all $y \in N(x)$, where $N$ is a function mapping an agent with its neighbours. \\
  \end{itemize}

  Example:
  \begin{figure}
    \centering
    \includegraphics[width=0.75\textwidth]{figures/sumo-rf-tls-triplet.png}
  \end{figure}

  
  \begin{columns}
    \begin{column}{0.4\textwidth}
      \centering
      {\footnotesize$R({TLS}_{1}) = f({TLS}_{1}) + f({TLS}_{2})$}
    \end{column}
    \begin{column}{0.6\textwidth}
      \centering
      {\footnotesize$R({TLS}_{2}) = f({TLS}_{2}) + f({TLS}_{1}) + f({TLS}_{3})$}
    \end{column}
  \end{columns}
\end{frame}

\section{Curriculum Learning}

\begin{frame}
\frametitle{Experience Engineering}
  \begin{figure}
    \centering
    \includegraphics[width=0.75\textwidth]{figures/curriculum-vs-monolithic.png}
  \end{figure}
\end{frame}

%[fragile]
% \begin{small}
%   \begin{verbatim}
%   TOK = (£|\*|~|%(,[0-9.]+)?(,[0-9.]+)?|[A-Z][A-Z0-9]+)
%   EXP = ^([0-9]+,)?([0-9]+,)?TOK(,TOK)*$
%   \end{verbatim}
% \end{small}

\section{Experiments and findings}

\begin{frame}
\centering
\Huge
Thank You
\end{frame}

\end{document}
